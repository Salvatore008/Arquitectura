\documentclass{article}
\usepackage[spanish]{babel}
\usepackage{amsmath}


\title{Formulas básicas de arquitectura}
\author{Luigi Quero}

\begin{document}
\maketitle

\begin{abstract}
Formulario de ecuaciones para calcular las prestaciones(rendimiento) del CPU y sus relaciones.
\end{abstract}

\tableofcontents

\section{Formulas están relacionadas con las prestaciones del CPU:}

\subsection{Prestación del CPU:}
$$
Prestaciones = \frac{1}{Tiempo\ de\ ejecuacion\ un\ programa}
$$

\subsection{Relación de las prestaciones entre equipos:}
$$
Prestaciones_X > Prestaciones_Y\\
$$
\\
$$
\frac{1}{Tiempo\ de\ ejecuacion_X} > \frac{1}{Tiempo\ de\ ejecuacion_Y} 
$$
\\
$$
Tiempo\ de\ ejecuacion_Y > Tiempo\ de\ ejecuacion_X
$$

\subsection{Relación cuantitativa de las prestaciones entre equipos:}
$$
\frac{Prestationes_X}{Prestaciones_y} = N 
$$
\\
$$
\frac{Prestationes_X}{Prestaciones_y} = \frac{Tiempo\ de\ ejecuacion_Y }{Tiempo\ de\ ejecuacion_X} = N
$$

\subsection{Medición de las prestaciones del CPU en base a otros factores:}
$$
Tiempo\ de\ ejecuacion\ de\ un \ programa = Ciclos\ para\ el\ programa * Periodo\ del\ reloj
$$
$$
Tiempo\ de\ ejecuacion\ de\ un \ programa = \frac{ Ciclos\ para\ el\ programa}{Frecuencia\ del\ reloj}
$$
\section{Ciclos e instrucciones del CPU como factores de prestaciones}

\subsection{Ciclos para un programa en base a prestaciones:}
$$
Ciclos\ de\ reloj\ del\ CPU\ para\ un\ programa = No\ de\ instrucciones * CPI
$$

\subsection{Ecuación clásica para medir las prestaciones de la CPU:}
$$
Tiempo\ de\ ejecucion = No\ de\ instrucciones\ * CPI * Periodo\ del\ reloj
$$
$$
Tiempo\ de\ ejecucion = \frac{No\ de\ instrucciones\ * CPI}{Frecuencia\ del\ reloj}
$$ 

\subsection{Ecuación elemental de prestaciones con factores simplificados:}
$$
Tiempo\ de\ ejecucion = \frac{segundos}{Instrucciones} = \frac{Instrucciones}{programa} * \frac{Ciclos de reloj}{Instrucciones} * \frac{Segundos}{Ciclos\ de\ reloj}
$$

\section{Ecuación para medir la potencia relativa de un procesador en
relación con otro:}
$$
\frac{Potencia_{nueva}}{Potencia_{antigua}} = \frac{Carga\ capacitiva_{nueva} * Voltaje^2_{nuevo} * Frecuencia_{nueva}}{Carga\ capacitiva_{antigua} * Voltaje^2_{antiguo} * Frecuencia_{antigua}}
$$

\section{Circuitos integrados y sus prestaciones en relación a su producción}

\subsection{Coste por dado de silicio:}
$$
Coste\ por\ dado = \frac{Area\ de\ la\ oblea}{Area\ del\ dado}
$$

\subsection{Porcentaje de dados sin defectos por tanda(aproximadamente:}
$$
Factor\ de\ produccion = \frac{1}{(1 (Defectos\ por\ area * Area\ del\ dado))^2}
$$

\section{Ley de Amdalh:}
$$
Tiempo\ de\ ejecucion\ mejorado = \frac{Tiempo\ de\ ejecucion\ por\ la\ mejora}{Cantidad de mejora} + Tiempo\ no\ afectado
$$
\section{MIPS(Millones de Instrucciones Por Segundo):}
$$
MIPS = \frac{No\ de\ instrucciones}{Tiempo\ de\ ejecucion * 10^6}
$$
\\
$$
MIPS = \frac{Frecuencia\ de\ reloj}{CPI * 10^6}
$$

\end{document}